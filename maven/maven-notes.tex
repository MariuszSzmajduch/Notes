\documentclass{report}
\title{maven-notes}

\usepackage[utf8]{inputenc}
\usepackage{color}
\usepackage{hyperref}

\begin{document}
\maketitle
\pagenumbering{gobble}
\tableofcontents




\part{POM}
\pagenumbering{arabic}


\chapter{Overview, Content}
Project Object Model, an XML representing of project resources. Shoud be \textbf{in the root directory of the project }it belongs to.


\section{Build Life-Cycles, Phases, Goals}
It's a tree:
\begin{itemize}
	\item \textbf{build life-cycles} contain:
	\item ...\textbf{phases}, they contain:
	\item \textbf{goals}
\end{itemize}

The interaction with a Maven is by sending a life-cycle, phase or goal name as a command. The command executes also \textbf{all predecessors}.


\section{Dependencies and Repositories}
JARs, libraries, local and remote repositories - It is one of the first Maven responsiblity to check them.


\section{Build Plugins}
Can be custom or predefined. Set of actions not covered by the standard Maven, but can be added  as plugins.


\section{Build Profiles}
They are used when a program need to be build in different ways. For instance testing and deployment version.



\chapter{POM Syntax}


\section{Elements}

Here are all elements of POM:
\begin{verbatim}
<project xmlns="http://maven.apache.org/POM/4.0.0"
         xmlns:xsi="http://www.w3.org/2001/XMLSchema-instance"
               xsi:schemaLocation="http://maven.apache.org/POM/4.0.0
         http://maven.apache.org/xsd/maven-4.0.0.xsd">

<modelVersion>4.0.0</modelVersion>

<!-- The Basics -->
<groupId>...</groupId>
<artifactId>...</artifactId>
<version>...</version>
<packaging>...</packaging>

<dependencies>...</dependencies>

<parent>...</parent>
<dependencyManagement>...</dependencyManagement>
<modules>...</modules>
<properties>...</properties>


<!-- Build Settings -->
<build>...</build>
<reporting>...</reporting>


<!-- More Project Information -->
<name>...</name>
<description>...</description>
<url>...</url>
<inceptionYear>...</inceptionYear>
<licenses>...</licenses>
<organization>...</organization>
<developers>...</developers>
<contributors>...</contributors>


<!-- Environment Settings -->
<issueManagement>...</issueManagement>
<ciManagement>...</ciManagement>
<mailingLists>...</mailingLists>
<scm>...</scm>
<prerequisites>...</prerequisites>
<repositories>...</repositories>
<pluginRepositories>...</pluginRepositories>
<distributionManagement>...</distributionManagement>
<profiles>...</profiles>
</project>
\end{verbatim}

\begin{itemize}
	\item modelVersion defines the version of POM. Version 4.0.0 is the only version supported by Maven 2 and 3 and is required.
	\item the minimum POM must contain:
	\begin{itemize}
		\item preamble - opennimg project tag with the atributes set as above
		\item groupId:artifactId:versionId fields, however \textbf{groupId and versionID can be inherited}
	\end{itemize}
\end{itemize}


\section{Example}
\begin{verbatim}
<project xmlns="http://maven.apache.org/POM/4.0.0"
                 xmlns:xsi="http://www.w3.org/2001/XMLSchema-instance"
                 xsi:schemaLocation="http://maven.apache.org/POM/4.0.0
                 http://maven.apache.org/xsd/maven-4.0.0.xsd">
    <modelVersion>4.0.0</modelVersion>

    <groupId>com.jenkov</groupId>
    <artifactId>java-web-crawler</artifactId>
    <version>1.0.0</version>
</project>content...
\end{verbatim}

\textbf{groupId}, \textbf{artefactId}, \textbf{version} tags, as well as \textbf{dots} are used to create directory structure for a project. For instance a JAR
file created using this POM would be placed in
\begin{verbatim}
MAVEN_REPO_dir/com/jenkov/java-web-crawler/1.0.0/java-web-crawler-1.0.0.jar
\end{verbatim}

\subsection{modelVersion}
Use \textbf{4.0.0} for Maven version 2 and 3.

\subsection{groupId}
Usage:
\begin{itemize}
	\item \textbf{project} name, usually \textit{the root java package} of the project, or
	\item \textbf{organisation} name
\end{itemize}
Maven repository matches its directory structure with the groupID. Each dot represents a subdirectory. In this example it is \textit{\textbf{MAVEN\_REPO}/com/jenkov}.
The MAVEN\_REPO is a path variable which is replaced by actuall path to Maven repository.

\subsection{artifactId}
It contains the name of the project. This name is also used as the name of JAR file produced when a project is built.

\subsection{version}
Nothing unusual.

\subsection{packaging}
\textbf{jar}, \textbf{maven-plugin}, also \textit{war}, \textit{ear}, \textit{rar}.When packaging is not defined then \textbf{jar is assumed} by Maven.



\chapter{Inheritance}


\section{Overview}
POMs can inherit from parent POM.


\section{Effective POM}
Show the Maven file which will be executed after \textbf{all inheritance is aplied}.
\begin{verbatim}
mvn help:effective-pom
\end{verbatim}




\part{Setting Files}



\chapter{Overview}

\begin{verbatim}
settings.xml
\end{verbatim}

They are used to define repository locations, profile files. Two standard locations (both otpional, however):
\begin{itemize}
	\item the \textbf{Maven installation directory}: \textit{\$M2\_HOME/conf/settings.xml}
	\item the \textbf{user's home directory}: \textit{\$\{user.home\}/.m2/settings.xml}
\end{itemize}




\part{Executing Maven}



\chapter{phase}
\begin{verbatim}
mvn phase1 phase2...
\end{verbatim}
...executes phases and \textbf{all their predecessors}. For instance:
\begin{verbatim}
mvn clean install
\end{verbatim}



\chapter{goal}
\begin{verbatim}
mvn phase:goal
\end{verbatim}




\part{Maven Directory Structure}



\chapter{Standard Directory Structure}
\begin{verbatim}
- src
    - main
        - java
        - resources
        - webapp
    - test
        - java
        - resources

- target
\end{verbatim}
If this structure is followed then there is no need to specify above directories. \textit{\textbf{src}} is the root directory of \textbf{source code} 
(in \textit{\textbf{main}}) and \textbf{test files} (in \textbf{\textit{test}}).

\textbf{\textit{target}} is created by Maven. It contains all output files (binaries, Jars, etc.). \textbf{\textit{mvn clean}} removes all files from this dear




\part{Dependency Management}



\chapter{Overview}
It is Maven built-in tool, which download external dependencies and also recursively all \textbf{transitive} dependencies in a depndance tree. They are downloaded
from the central Maven repository, however, only when not present in the local repository. If a dependency is missing in the central repository
(\textbf{external}), it can be downloaded and added
anually. \textbf{The directory structure must much POM!} See syntax below.



\chapter{Syntax}
Each dependency is described by its:
\begin{itemize}
	\item groupId
	\item artifactId
	\item version
\end{itemize}


\section{Elements}
\begin{verbatim}
<dependency>
    <groupId>junit</groupId>
    <artifactId>junit</artifactId>
    <version>4.0</version>
    <type>jar</type>
    <scope>test</scope>
    <optional>true</optional>
</dependency>}
\end{verbatim}
\begin{itemize}
	\item \textbf{groupID, artifactId, version} - dependency coordinates
	\item \textbf{type} - correspond to depndant \textbf{packaging} type. It also \textbf{defaults to jar}.
\end{itemize}


\section{Example}

\begin{verbatim}
<dependencies>

    <dependency>
        <groupId>org.jsoup</groupId>
        <artifactId>jsoup</artifactId>
        <version>1.7.1</version>
    </dependency>

    <dependency>
        <groupId>junit</groupId>
        <artifactId>junit</artifactId>
        <version>4.8.1</version>
        <scope>test</scope>
    </dependency>

</dependencies>
\end{verbatim}

This POM will result with following directories:
\begin{verbatim}
MAVEN_REPOSITORY_ROOT/junit/junit/4.8.1

MAVEN_REPOSITORY_ROOT/org/jsoup/jsoup/1.7.1
\end{verbatim}



\chapter{External Dependencies}
They are the dependencies not present in Maven repositories (neither local, central or remote).


\section{Syntax Example}
\begin{verbatim}
<dependency>
    <groupId>mydependency</groupId>
    <artifactId>mydependency</artifactId>
    <scope>system</scope>
    <version>1.0</version>
    <systemPath>${basedir}\war\WEB-INF\lib\mydependency.jar</systemPath>
</dependency>

\end{verbatim}

\begin{itemize}
	\item \textbf{groupId} and \textbf{artifactId} are set to the name of the dependency
	\item standard \textbf{scope} s \textit{system}
	\item \textbf{systemPath} points to the location of the JAR file.
	\begin{itemize}
		\item \textbf{\$\{basedir\}} is a variable storing the path to the POM
	\end{itemize}
\end{itemize}



\chapter{Snapshot Dependencies}
When set then Maven always downloads the latest version of the dependency. The frequency of updates checking is configurable.
Snapshot can be set:
\begin{itemize}
	\item for \textbf{entire project} at the beginning of a POM
	\begin{verbatim}
	<version>1.0-SNAPSHOT</version>
	\end{verbatim}
	
	\item for a \textbf{particular dependency}
\end{itemize}




\part{Maven Repositories}



\chapter{Overview}
Three types:
\begin{itemize}
	\item local
	\item central
	\item remote
\end{itemize}
When maven looks for dependencies then looks for them in above directories in this order. Each dependency is a JAR file with associated POM, which contains 
the information of further dependencies. Ths allows to download recursively entire dependency tree.

\chapter{Local Repository}
This is a directory on the developer's pc. It contains all dependencies Maven has downloaded. Each dependency is downloaded only once, then it is shared between projects, if needed. Default location of the repository can be changed by updating settings.xml, tag \textit{\textless localRepository\textgreater}.



\chapter{Central Repository}
This community maintained repository. Maven looks it up when a dependency is not present in the local repository. No configuration is required.


\chapter{Remote Repository}
Maintained anywhere by a web server. Usually used to host projects internal to an organisation.




\part{Build Life-Cycles, Phases, Goals}



\chapter{Build Life-Cycles}
There are 3 built-in
\begin{enumerate}
	\item default
	\item clean
	\item site
\end{enumerate}

\paragraph{default} deals with compilation and packaging. \textbf{It cannot be executed directly!} Some contained phase or goal has to be invoked instead. Examples of predefined phases:
\begin{itemize}
	\item \textit{validate} - checks a project integrity (all dependencies downloaded, project is correct)
	\item \textit{compile}
	\item \textit{test}
	\item \textit{package} - generates a JAR file
	\item \textit{install} - installs a project into the \textbf{local repository} for use as a \textbf{dependency} for other local projects.
	\item \textit{deploy} - copies the project to \textbf{the remote repository} for sharing with other developers and projects.
\end{itemize}

\paragraph{clean} removes all files from \textit{target} directory

\paragraph{site} generates a documentation.




\part{IDE Support}



\chapter{Eclipse}


\section{Creating a Simple Project}
\begin{itemize}
	\item 'new'
	\item 'other'
	\item 'Maven project'
	\item on "New Maven Project' screen \textbf{check 'Create simple project'} in order to skip the window with artifact selection.
	\item provide Maven coordinates and \textit{finish}.
\end{itemize}





\part{Appendix}



\chapter{Commands}


\section{Executing Maven}

\subsection{phase}
\begin{verbatim}
mvn phase1 phase2...
\end{verbatim}
...executes phases and \textbf{all their predecessors}. For instance:
\begin{verbatim}
mvn clean install
\end{verbatim}

\subsection{goal}
\begin{verbatim}
mvn phase:goal
\end{verbatim}


\section{Version}
\begin{verbatim}
mvn --version
\end{verbatim}

or

\begin{verbatim}
mvn -v
\end{verbatim}


\section{Effective POM}
\begin{verbatim}
mvn help:effective-pom
\end{verbatim}

\section{Clean Up}
\begin{verbatim}
mvn clean
\end{verbatim}
...removes all files from \textit{target} directory.



\chapter{Vocabulary}


\section{P}

\subsection{POM}
Project Object Model, an XML representing of project resources. Shoud be in the root directory of the project it belongs to.



\chapter{Resources}


\section{Web}
\begin{itemize}
	\item \url{http://maven.apache.org/pom.html#Maven_Coordinates}
	\item \url{http://tutorials.jenkov.com/maven/maven-tutorial.html}
	\item \url{https://maven.apache.org/guides/index.html}
\end{itemize}

\end{document}