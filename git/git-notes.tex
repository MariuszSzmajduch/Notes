\documentclass{report}
\title{git-notes}

\usepackage[utf8]{inputenc}
\usepackage{listings}
\usepackage{color}
\usepackage{hyperref}


\begin{document}
\maketitle
\pagenumbering{gobble}
\tableofcontents

\chapter{Help}
\pagenumbering{arabic}

\section{CLI}
\begin{verbatim}
git <command> -h
\end{verbatim}



\chapter{Architecture}
\begin{itemize}
	\item working directory\\
           the 'visible' part which lets do modification
		\begin{itemize}
			\item \textbf{staging index}\\
				'waiting area' - mediates between working area and local repo. Initially in sync with the repo, 
				but allows to store changes done in working area and then to send the changes to local repo 'in-one-go'\\
				Can be behind the repo after some new content has been fetched.
	
			\item \textbf{stashing area}\\
			'a pocket', intended to store changes that haven't been commited. useful when we need to switch a branch, but the changes are not ready to commit yet.
		\end{itemize}

	\item local repo	
	this is a project tree stored locally. Two pointers operate on it:
\begin{itemize}
\item \textbf{HEAD}\\
points to the commit which copy is present in the working directory
\item \textbf{remote\_repo/remote\_branch}\\
usually 'origin/master' - points to the last commit fetched from a remote repo 
\end{itemize}

\item remote repo
\end{itemize}

\chapter{Configuration}
\section{config ranges}
\begin{itemize}
\item System
\begin{lstlisting}
git config --system
\end{lstlisting}

\item User
\begin{lstlisting}
git config --global
\end{lstlisting}

\item Project
\begin{lstlisting}
git config
\end{lstlisting}

\item edit examples
\begin{lstlisting}
git config --global user.email "abc@mail.com"
git config --global core.editor "vim"
\end{lstlisting}
\end{itemize}

\section{configurable properties}
\begin{itemize}

\item user
\begin{itemize}
\item \begin{verbatim}
user.name=
\end{verbatim}
\item \begin{verbatim}
user.email=
\end{verbatim}
\end{itemize}

\item core
\begin{itemize}
\item \begin{verbatim}
core.editor=
\end{verbatim}
\item \begin{verbatim}
core.excludesfile=
\end{verbatim}
\end{itemize}

\item color\begin{verbatim}
color.ui=
\end{verbatim}

\item remote URLs\\
look at 'Remote' chapter, 'Managing URLs' section

\end{itemize}

\section{config listing}
\begin{itemize}
\item all
\begin{lstlisting}
git config --list
\end{lstlisting}

\item specific
\begin{lstlisting}
git config user.email
\end{lstlisting}
\end{itemize}

\section{customised prompt}
\subsection*{UNIX-like}
\begin{quote}
\begin{verbatim}
export PS1='\W$(__git_ps1 "($s)") > '
\end{verbatim}
add it to .bashrc (.bash\_profile):
\begin{lstlisting}
# uncomment current prompt export if exist
# export PS1="some_user"
....
# add at the end of the script
export PS!='\W$(__git_ps1 "($s)") > '

\end{lstlisting}
\end{quote}

\subsection*{Windows}
\begin{quote}
similar format should be preconfigured. to get exact the same:
\begin{verbatim}
export PS1='\W$(__git_profile "(%s)") > '
\end{verbatim}
save as \textbf{.bash\_profile} in \textbf{/Users/current\_user} dir
\end{quote}

\section{aliases}
\begin{verbatim}
git config --<scope> alias.<abbreviation> original_command
\end{verbatim}
put original command between double quotes if it contains space(s)


\chapter{Common Tasks}
\section{diff}
\begin{itemize}
\item \begin{verbatim}
git diff <file_name>
\end{verbatim}
compares \textcolor{red}{\textbf{working}} directory against \textcolor{red}{\textbf{staging}} area - for each file, on line-by-line basis
\item \begin{verbatim}
git diff --staged <file_name> (OBSOLETE: git diff --cached)
\end{verbatim}
compares \textcolor{red}{\textbf{staging}} against \textcolor{red}{\textbf{repository}}

\item \begin{verbatim}
git diff <SHA>
\end{verbatim}
compares state in a particular commit wit the current state in a working dir

\item \begin{verbatim}
git diff <SHA>..<SHA>\end{verbatim}
compares two particular commits

\item use switches to change the way the differences are displayed
\begin{lstlisting}
git diff --color-words <file_name>
\end{lstlisting}
\end{itemize}

\section{show}
\begin{quote}
For commits it shows the log message and textual diff. It also presents the merge commit in a special format as produced by git diff-tree --cc. (git doc, \url{https://git-scm.com/docs/git-show} )
\end{quote}

\section{delete}
\begin{itemize}
\item \begin{verbatim}
git rm file_name
\end{verbatim}
if present - removes definitely the file from working directory (it doesn't go to the trash bin)\\
stages 'delete' operation
\end{itemize}

\section{rename/move}
\begin{itemize}
\item \begin{verbatim}
git mv old_name new_name
\end{verbatim} 
\end{itemize}

\section{commit}
\begin{itemize}
\item \begin{verbatim}
commit -a
\end{verbatim}
combines 'add' and 'commit'. Ignores deleted and new files.
\end{itemize}

\section{undo}
\begin{itemize}
\item in \textbf{working directory}
\begin{itemize}
\item\begin{verbatim}
git checkout <commit_SHA> <branch_name> <file_name>
\end{verbatim}

\item \begin{verbatim}
git checkout -- file_name
\end{verbatim} 
to redo to the same state as at the pointer in repository

\item \begin{verbatim}
git checkout file_name
\end{verbatim}
branch name not required if it is current one and there is no branch with the same name as the file

\item \begin{verbatim}
git checkout SHA_number <branch_name> <file_name>
\end{verbatim}
sets both \textcolor{red}{\textbf{working directory}} and \textcolor{red}{\textbf{stage area}} to the state in a particular commit identified by SHA

\item \begin{verbatim}
 git revert SHA_number
\end{verbatim}
reverts all changes done in a particular commit - updates working directory and makes \textcolor{red}{\textbf{new commit}} (this can be switched out) whith those reverted changes. \textbf{revert} is for simple changes, \textbf{merge} for complex. 
\begin{itemize}
\item \begin{verbatim}
git revert SHA_number --in
\end{verbatim}
doesn't make commit. allows to append further modification to next commit
\end{itemize}

\item \begin{verbatim}
git reset ...
\end{verbatim}
look at 'reset' subsection
\end{itemize}

\item in \textbf{staging area}
\begin{verbatim}
git reset HEAD <file_name>
\end{verbatim}
removes all changes when no file provided. look 'reset' section for more about reset.

\item in \textbf{repository}
\begin{verbatim}
git commit --amend <-m "...">
\end{verbatim}
only \textcolor{red}{\textbf{last commit}} editable! it appends staged changes to current commit, the message can be changed as well
\end{itemize}

\section{git reset HEAD ...}
\begin{quotation}
'Rewind' - sets the head on a particular commit; consecutive commits and logs becomes invisible. They can be accessed only by their SHA.
\end{quotation}
\begin{itemize}
\item \begin{verbatim}
git reset --soft ...
\end{verbatim}
sets pointer to new position, makes no changes to working dir and index (staging area)

\item \begin{verbatim}
git reset ..., git reset --mixed ...
\end{verbatim}
default mode. sets the pointer; sets staging index to match repository at the pointer. working directory stays intact

\item \begin{verbatim}
git reset --hard
\end{verbatim}
sets the pointer; sets both working dir and staging index to match repository at the pointer. later changes, commits, are lost
\end{itemize}

\section{HEAD in detached mode}
This means that the HEAD pointer is on some particular commit \textbf{outside} any branch. It could've happen after particular commit was checked out using its SHA1. To attach back \textbf{checkout} any branch.

\section{untracked files - delete}
\begin{itemize}
\item \begin{verbatim}
git clean -n
\end{verbatim}
tests, which files will be deleted
\item \begin{verbatim}
git clean -f
\end{verbatim}
destructive command - permanently removes untracked files
\end{itemize}

\section{.gitignore}
\begin{itemize}
\item syntax
\begin{verbatim}
* ? [abc] [a-c1-6] !
\end{verbatim}
\# starts a comment line; blank lines are ignored

\item project scope of ignore
\begin{quote}
create and edit .gitignore (without extention) in the repository root
\end{quote}

\item per-user ignore
\begin{verbatim}
git config --global core.excludesfile <file_path>
\end{verbatim}
to tell where .gitignore file is, the edit the file. typical filepath:
\begin{verbatim}
~/.gitignore_global  <- Linux
/Users/user_name/.gitignore  <- Windows
\end{verbatim}
\end{itemize}

\section{untracking files}
\begin{quote}
to stop tracking the  file by staging index:
\begin{verbatim}
git rm --cached <file_name>
\end{verbatim}
then add the file to \textbf{.gitignore} in order to prevent git to ask to add the file in the future.\\
the file is still kept in \textcolor{red}{\textbf{working dir}} and \textcolor{red}{\textbf{repository}}, but it isn't tracked for further changes by staging index
\end{quote}

\section{referencing ancestor commits}
\begin{quote}
for instance from HEAD:
\begin{verbatim}
HEAD~ = HEAD~1 = HEAD^
HEAD~2 = HEAD^^
HEAD~3 = HEAD^^^
\end{verbatim}
\end{quote}

\section{content listing}
\begin{quote}
\begin{verbatim}
git ls-tree <tree-ish(es)>
\end{verbatim}
tree-ish - branch (last commit), SHA, tag, dir
\end{quote}

\section{git log}
\begin{itemize}

\item my preferred
\begin{verbatim}
git log --online --graph --all --decorate
\end{verbatim}

\item short
\begin{verbatim}
git log --oneline
\end{verbatim}

\item tree of branches
\begin{verbatim}
git log --graph
\end{verbatim}

\item detailed
\begin{verbatim}
git log -p
\end{verbatim}
shows changes as shown by \textbf{diff}  command

\item particular tree-ish in detail
\begin{verbatim}
git show <tree-ish>
\end{verbatim}
shows:
\begin{itemize}
\item \textbf{content} of files, directories
\item particular \textbf{commits} like by "log -p"
\end{itemize}

\item time
\begin{verbatim}
git log --since=".." --until="..."
\end{verbatim}

\item some popular:
\begin{verbatim}
git log --grep="...."
git log --author="..."
\end{verbatim}

\item more at git help 

\end{itemize}

\section{stashing}
\begin{itemize}
\item saving in stash
\begin{verbatim}
git stash save "some message"
\end{verbatim}
like \textbf{git reset --hard HEAD}, but changes are stashed

\item listing a stash 
\begin{verbatim}
git stash list
\end{verbatim}
\textcolor{red}{\textbf{stash@\{0\}}}: \textless branch\_name\textgreater: ".." - stash reference

\item showing changes saved in stash
\begin{verbatim}
git stash show -p <stash_ref>
\end{verbatim}
Shows what changes this stash would apply\\
\textbf{stash is not bound to any commit. can be taken from one working dir and aplied to some other working dir}

\item applying changes from stash
\begin{verbatim}
git stash pop
git stash apply
\end{verbatim}
apply changes to current working dir;\\
\textbf{pop} drops stash, \textbf{apply} allows multiple use

\item removing from stash
\begin{itemize}
\item \begin{verbatim}
git stash drop stash@{id}
\end{verbatim}

\item \begin{verbatim}
git stash clear
\end{verbatim}
removes all

\end{itemize}
\end{itemize}

\chapter{Branches}

\section{new branch}
\begin{itemize}

\item \begin{verbatim}
git branch branch_name
\end{verbatim}
\textcolor{red}{\textbf{git branch...}} creates new branch

\item \begin{verbatim}
git checkout -b branch_name
\end{verbatim}
\textcolor{red}{\textbf{git checkout -b...}} creates new branch and switches to it
 
\end{itemize}

\section{delete}
\begin{quote}
Must not be current branch
\begin{itemize}
\item \begin{verbatim}
git branch -d branch_name
\end{verbatim}
works for \textcolor{red}{\textbf{fully merged}} branches only

\item \begin{verbatim}
git branch -D branch_name
\end{verbatim}
works for \textcolor{red}{\textbf{unmerged}} branches, too
\end{itemize}

\end{quote}


\section{list of branches}
\begin{itemize}
\item \begin{verbatim}
git branch
\end{verbatim}
lists \textbf{local} branches

\item \begin{verbatim}
git branch -r
\end{verbatim}
lists \textbf{remote} branches

\item \begin{verbatim}
git branch -a
\end{verbatim}
lists \textbf{all} branches (local + remote)
\end{itemize}

\section{switching a branch}
\begin{quote}
\begin{verbatim}
git checkout branch_name
\end{verbatim}
\begin{itemize}
\item \textit{'swapping context'}
\item working dir should be \textit{clean} - all modifications should be committed, stashed or discarded
\end{itemize}
\end{quote}

\section{rename}
\begin{quote}
\begin{verbatim}
git branch -m old_name new_name
\end{verbatim}
\textcolor{red}{\textbf{-m}! Not \textbf{-mv}!}
\end{quote}

\section{merging}
\subsection{merge}

\begin{enumerate}
\item ensure this is a destination branch
\item ensure the working directory is clean
\item \begin{verbatim}
git merge <source_branch>
\end{verbatim}
\end{enumerate}

\begin{itemize}
\item 
\begin{verbatim}
git branch --merged
\end{verbatim}
returns a list of fully incorporated branches - they can be merged \textcolor{red}{\textbf{fast-forward (ff-merge)}}

\item \begin{verbatim}
git merge --no-ff <branch_name>
\end{verbatim}
created merge commit even if it is ff-merge

\item \begin{verbatim}
git merge --ff-only <branch_name>
\end{verbatim}
merges only if ff-merge is possible; aborts otherwise

\end{itemize}

\subsection{resolving conflicts}
\begin{itemize}
\item abort
\begin{verbatim}
git merge --abort
\end{verbatim}

\item resolve manually
\begin{itemize}
\item open files, find conflict spots, manually fix them; useful:
\begin{verbatim}
git show <object>
\end{verbatim}
, and put SHA1 as an object; look section 2.2
\item stage modified files
\item commit
\begin{itemize}
\item this is merge commit - merge completed!
\item message unnecessary
\end{itemize}
\end{itemize}

\item resolving using tools
\begin{quote}
\begin{verbatim}
git mergetool --tool=...
\end{verbatim}
type \textbf{git mergetool} to get list of available/recommended tools; a tool can be added to the config file
\end{quote}

\end{itemize}

\chapter{Remotes}
\section{origin/master}
\begin{itemize}
\item this is a pointer to last fetched commit.
\item need to be in sync with local and remote master before push:
\begin{itemize}
\item fetch (or pull) to sync with a remote
\item merge locally to resolve conflicts and sync locally (master and origin/master pointing to the same commit)
\item repeat fetch+merge (or pull) until all conflicts are resolved and all syncs established; this makes ff-merge of remote master and master/origin possible

\end{itemize}
\end{itemize}

\section{Managing URLs}
\subsection{list}
\begin{quote}
\begin{itemize}

\item \begin{verbatim}
git remote
\end{verbatim} 
returns remote identifiers

\item \begin{verbatim}
git remote -v
\end{verbatim}
detail info including URLs

\end{itemize}
\end{quote}

\subsection{add}
\begin{quote}
\begin{verbatim}
git remote add <remote_name> <URL>
\end{verbatim}
it's a convention to call primary remote \textcolor{red}{\textbf{origin}}.\\
\begin{itemize}
\item https URL:
\begin{verbatim}
https://github.com/<user_name>/<repo_name>.git
\end{verbatim}

\item ssh URL:
\begin{verbatim}
git@github.com:<user_name>:<repo_name>.git
\end{verbatim}
\end{itemize}
\end{quote}

\subsection{remove}
\begin{quote}
\begin{verbatim}
git remote -rm <remote_name>
\end{verbatim}
\begin{quote}
remote name as listed by
\begin{verbatim}
git remote -v
\end{verbatim}
\end{quote}
\end{quote}

\section{Collaboration}
\subsection{create local and remote repositories}
\subsubsection{remote repo from local one}

\begin{enumerate}
\item \begin{verbatim}
git init
\end{verbatim}
in the root dir of a project
\item \begin{verbatim}
git remote add <remote_repo_name> <URL>
\end{verbatim}
look at Remotes $\rightarrow$ add
, sec. 4.2.2
\item \begin{verbatim}
git push -u origin master
\end{verbatim}
pushes local content to remote repo; -u makes local and remote branches in sync; look at Remotes $\rightarrow$ send
\end{enumerate}

\subsubsection{clone remote repo to local repo}
\begin{itemize}
\item \begin{verbatim}
git clone <remote_URL>
\end{verbatim}
\begin{itemize}

\item creates local folder using remote repo name

\item clones remote project to the folder
\end{itemize}

\item \begin{verbatim}
git clone <remote_URL> <folder_name>
\end{verbatim}

\begin{itemize}
\item creates local folder with the specified name

\item clones remote project to the folder
\end{itemize}
\end{itemize}

\subsection{send}
It works only if ff-merge is possible on the remote side. \\
Look at Remote $\rightarrow$ origin/master.
\begin{itemize}
\item \begin{verbatim}
git push <remote_repo> <remote_branch>
\end{verbatim}
usually:
\begin{verbatim}
git push origin master
\end{verbatim}
or
\begin{verbatim}
git push
\end{verbatim}
if current branch is tracked

\item
\begin{verbatim}
git push -u ...
\end{verbatim}
also sets a local branch to track a remote

\end{itemize}

\subsection{receive}
\begin{itemize}
\item \begin{verbatim} git fetch + git merge
\end{verbatim}
\begin{itemize} 
\item merge works exactly the same as for any other merge
\item \begin{verbatim}
git pull
\end{verbatim}
does exactly the same if ff-merge is possible (no conflicts)
\end{itemize}

\item \begin{verbatim}
git fetch <remote_name>
\end{verbatim}
\begin{itemize}
\item we can omit remote name if there's one remote only
\item Non-destructive!
\item updates origin/master, synchronises with remote repo
\item origin/master doesn't reflect current state of a remote repo, it's only a copy of the last fetched state.
\item fetch doesn't do any changes neither to local repo, nor staging area, nor local working dir

\end{itemize}

\item merge with origin/master the same way as with any other branch
\begin{itemize}
\item \begin{verbatim}
git show origin/master
\end{verbatim}
shows what was fetched
\item \begin{verbatim}
git diff master..master/origin
\end{verbatim}
shows changes to apply locally

\end{itemize}

\end{itemize}

\subsection{remote branches}
\begin{itemize}
\item list
\begin{itemize}

\item \begin{verbatim}
git branch -r
\end{verbatim}
remote only

\item \begin{verbatim}
git branch -a
\end{verbatim}
 all
\end{itemize}

\item create
\begin{verbatim}
git branch local_branch_name remote_name/remote_branch_name
git checkout -b local_branch_name remote_name/remote_branch_name
\end{verbatim}
\begin{itemize}
\item creates local and remote branch at the same time
\item make the local one tracked and in sync with the remote one
\item \textbf{git checkout -b...} also switches to this new branch 
\end{itemize}

\item delete
\begin{quote}
\begin{verbatim}
git push origin --delete remote_branch_name
\end{verbatim}
or (older version)
\begin{verbatim}
git push origin :remote_branch_name
\end{verbatim}
(push 'nothing' to a remote branch)
\end{quote}

\end{itemize}

\chapter{GitHub}
\section{help}

\begin{itemize}

\item GitHub doc:\\
\url{file:///D:/IT/version%20control/git/web/GitHub%20Help.htm}

\item Short github tutorials:\\
\url{file:///D:/IT/version%20control/git/web/GitHub%20Guides.htm}

\item Customizing GitHub Pages
\url{https://help.github.com/categories/customizing-github-pages/}
\end{itemize}

\section{GitHub pages}
To create and host web pages based on GitHub repos:\\
\url{https://guides.github.com/features/pages/}



\chapter{Resources}
\begin{itemize}
	\item official documentation:\\
	\url{https://git-scm.com/docs}
\end{itemize}

\end{document}
