\documentclass{report}
\title{Deployment Notes}
\author{msz}

\usepackage[utf8]{inputenc}
\usepackage{color}
\usepackage{hyperref}
\usepackage{breakurl}

\usepackage{tabto}

\begin{document}
\tableofcontents




\part{Practices}




\part{Ant}



\chapter{Overview}
\textbf{Another Neat Tool.} Open and portable standard to automate 
build and deploy processes. Build files are in XML format. By convention 
they are called \textit{build.xml}.



\chapter{Build File}


\section{overview and a document structure}
By convention they are called \textit{build.xml}. Structured wit following
tags:
\begin{itemize}
  \item \textbf{preamble} (optional)
  \item \textbf{project} - mandatory!
  \item \textbf{target} - at least 1 mandatory
\end{itemize}


\section{preamble}
\textbf{Optional!}

  \begin{verbatim}
  <?xml version = "1.0" ?>
  \end{verbatim}
  
  There should be no tab, blank line or whitespace before!


\section{tags}

\subsection{project}
This element is required! Attributes:
\begin{itemize}
  \item name (optional)
  \item default (\textbf{mandatory!}). It specifies which target should be executed when command \textit{ant} is not suplied
  with an argument. 
  \item basedir; optional - location of the build file is taken as a root dir
  if not specified
  \item xmlns:ivy=``antlib:org.apache.ivy.ant defines an XML namespace for ivy. Mandatory in order to resolve dependencies with ivy.
\end{itemize}

\subsection{target}
\begin{verbatim}
ant -p
\end{verbatim}
lists all available targets.
\textbf{At least one is mandatory!} Attributes:
\begin{itemize}
  \item name = mandatory
  \item depends - optional; points to other targets that have to be execute as a perequisites in order to 
  execute this target
  \item description
  \item if; a condition that has to be true in order to execute the task.
  \item unless; adds the task to the dependency list of the specified Extension Point.
\end{itemize}

\subsection{task}
It is a named tag that actually does something. There is about
150 built-in tasks.


\section{list of tasks}
\url{http://ant.apache.org/manual/index.html}

\subsection{delete}
Deletes from a file system. Attributes:
\begin{itemize}
  \item dir - \textbf{mandatory}.
\end{itemize}

\subsection{echo}
Prints to stdout.

\subsection{ivy:retrieve}
Executes ivy script. Needs XML namespace declaration in the project tag.
\begin{verbatim}
<project xmlns:ivy = "antlib:org.apache.ivy.ant" ...
    ...
    <target name = "resolve" ...>
        <ivy:retrieve />
    </target>
    ...
</project>
\end{verbatim}

\subsection{java}
Runs java application. Example running executable jar:
\begin{verbatim}
<java jar = "dir-to/file.jar" fork = "true"/>
\end{verbatim}

\subsection{javac}
Attributes:
\begin{itemize}
  \item srcdir - \textbf{mandatory}
  \item destdir \textbf{mandatory}
\end{itemize}

\subsection{jar}
Packages an application. For startable jar-package additional nested task \textit{manifest} is required
Standard usage of \textit{basedir}. Exmaple:
\begin{verbatim}
<jar destfile = "dir-to/file.jar" basedir = "root-of/java.class">
    <manifest>
        <attribute name = "Main-Class" value = "package.address.of.MainClass"\>
    </manifest>
</jar>
\end{verbatim}

\subsection{javadoc}

\subsection{mkdir}
Attributes:
\begin{itemize}
  \item dir \textbf{mandatory}
\end{itemize}

\section{ant properties}

\begin{verbatim}
<property name = "prop_name" value = "prop_value"/>
\end{verbatim}

\begin{itemize}
  \item There are 10 predefined, 
  they can be also custom specified. 
  \item \textbf{scope} - global.
  \item \textbf{lifecycle} - the lifecycle of immediate surrounding tag.
  If it is a project - then the property is set for all tasks. If it is within a task - the property
  is set globally, but just for this particular task.
  \item \textbf{modification and accessibility}
  	\begin{itemize}
  	  \item they are immutable
  	  \item they value of any property can be used anywhere in the code using
  	  following syntax:
  	  \begin{verbatim}
  	  ${prop_name}
  	  \end{verbatim}
  	  See: BuildFile.tags.tasks.ivy:retrieve
  	  \end{itemize}
\end{itemize}

\subsection{predefined ant properties}

\begin{itemize}
  \item \textbf{ant.file}\\
The full location of the build file.

	\item \textbf{ant.version}\\
The version of the Apache Ant installation.

	\item \textbf{basedir}\\
The basedir of the build, as specified in the basedir attribute of the project element.

	\item \textbf{ant.java.version}\\
The version of the JDK that is used by Ant.

	\item \textbf{ant.project.name}\\
The name of the project, as specified in the name atrribute of the project element.

	\item \textbf{ant.project.default-target}\\
The default target of the current project.

	\item \textbf{ant.project.invoked-targets}\\
Comma separated list of the targets that were invoked in the current project.

	\item \textbf{ant.core.lib}\\
The full location of the Ant jar file.

	item \textbf{ant.home}\\
The home directory of Ant installation.

	\item \textbf{ant.library.dir}\\
The home directory for Ant library files - typically ANT\_HOME/lib folder.

\end{itemize}


\section{ant-data types}
They are predefined tags. They are kind of built-in services.

\subsection{fileset}
Is used as a filter to include or exclude files that match a particular pattern.
Attributes:
\begin{itemize}
  \item id
  \item dir - uri of the root folder
  \item casesensitive
\end{itemize}

Tags:
\begin{itemize}
  \item include
  \item exclude
\end{itemize}
They have an attribute \textit{name}.

\begin{verbatim}
<fileset dir = "${some_dir}" casesensitive = "yes">
	<include name = "**/*.java"/>
	<exclude name = "**/*.class"/>
\end{verbatim}

\subsection{filelist}
Explicit list of files. 
\begin{itemize}
  \item no wild cards
  \item can be applid to non-existing files, too 
\end{itemize}
\textit{fileset} \textbf{filters}
files, \textit{filelist} names them explicitly:
\begin{verbatim}
<filelist id = "list_id" dir = "${base_url}">
	<file name = "file1_name"\>
	<file name = "file2_name"/>
	...
</filelist>
\end{verbatim}

\subsection{patternset}
Meta characters:
\begin{itemize}
  \item ? - exactly one character
  \item * - zero or more characters
  \item ** - zero ore more dirs
\end{itemize}

Tags:
\begin{itemize}
  \item id - pattern identifier
  \item include, exclude - pattern definitions
  \item refid - actual use of pattern definition
\end{itemize}

\textbf{Example}. First a pattern is specified:
\begin{verbatim}
<patternset id = "java.src">
	<include name = "**/*.java"/>
	<exclude name = "**/*.class"/>
</patternset>
\end{verbatim}


Then the pattern can be reused by its id:
\begin{verbatim}
<fileset dir = "${src} casesensitive = "yes">
	<patternset refid = "java.src"/>
</fileset>
\end{verbatim}



\chapter{Property File}
By convention it is \textit{build.properties} or \textit{build.properties.ver},
where \textit{ver} specifies a version, like \textit{prod}, \textit{test}.
Should be placed at the same location as \textit{build.xml} file.
Contains a list of key-value pairs \textbf{without quotes!}. Example:

\begin{verbatim}
# My Project Details
location = local machine
buildversion = 4.2
\end{verbatim}

Then we can use properties stored in the property file in \textit{build.xml}
by specyfying the value of \textit{file} attribute in an ant property tag:
\begin{verbatim}
<property file = "build.property.test"/>
\end{verbatim}

\textbf{Comments:}
\begin{verbatim}
# This is a comment!
# ...and this is another comment
location = my local machine
buildversion = 4.2
\end{verbatim}


\chapter{Ant in Eclipse}
\url{https://codesjava.com/ant-create-java-document-in-eclipse}


\chapter{Resources}

\begin{itemize}
  \item \url{https://www.tutorialspoint.com/ant/index.htm}
\end{itemize}




\part{Ivy}



\chapter{Overview}
\begin{itemize}
  \item resolves dependencies
  \item in XML
  \item maven2 repository
  \item \textit{ivy.xml} - conventional config file name
  \item See: Ant.BuildFile.tags.tasks.ivy:retrieve for syntax how to refer from ant
  build file.
\end{itemize}



\chapter{Syntax}


\section{structure}
Mandatory tags in bold, mandatory attributes in bold-red
\begin{itemize}
  \item \textbf{ivy-module} \textbf{\textcolor{red}{version}}
  \begin{itemize}
    \item \textbf{info} \textbf{\textcolor{red}{organisation module}}
    \item \textbf{dependencies}
    \begin{itemize}
      \item \textbf{dependency} \textbf{\textcolor{red}{org name rev}}
      \item \ldots
    \end{itemize}
  \end{itemize}
\end{itemize}


\section{ivy-module}

\begin{verbatim}
<ivy-module version = "2.0">
   ...
<\ivy-module>
\end{verbatim}

\textbf{Mandatory!} It's a preamble that states that this is ivy-file. 
Attributes:
\begin{itemize}
  \item version \textbf{mandatory!}.
\end{itemize}


\section{info}

\textbf{Mandatory}. Custom values that identify the project. Attributes \textbf{(mandatory!)}:
\begin{itemize}
  \item organisation - source owner, creator\ldots
  \item module - usually application name or its part.
\end{itemize}

\begin{verbatim}
<info organisation = "msz" module = "my-app"\>
\end{verbatim}

\section{dependencies}

Contains set of \textit{dependency} tags.

\subsection{dependency}
Seperate entry for each dependency. Attributes (\textbf{mandatory}):
\begin{itemize}
  \item org
  \item name
  \item rev
\end{itemize}
Value for each attribute can be found here:\\
\url{https://mvnrepository.com/}\\
Information provided as POM. The way to convert:
\begin{itemize}
  \item groupID - org
  \item artifactId - name
  \item version - rev
\end{itemize}




\chapter{Resources}

\url{http://ant.apache.org/ivy/history/2.3.0-rc2/tutorial.html}




\part{Graddle}



\chapter{Overview}

\section{Project}
This interface is the main API you use to interact with Gradle
from your build file. From a Project, you have programmatic access 
to all of Gradle's features. During build initialisation, Gradle assembles a
Project object for each project which is to participate in the build.
A project is essentially a collection of Task objects. Each task performs some
basic piece of work, such as compiling classes, or running unit tests, or zipping up a WAR file



\chapter{Resources}
Official getting started tutorial:\\
\url{https://guides.gradle.org/creating-new-gradle-builds/}

\end{document}