\documentclass{report}
\title{latex-notes}

\usepackage[utf8]{inputenc}
\usepackage{color}
\usepackage{hyperref}
\usepackage{tabto}

\begin{document}
\pagenumbering{gobble}
\tableofcontents

\pagenumbering{arabic}




\part{Resources}

\begin{itemize}
	\item UML Distilled textbook by Martin Fowler 
	\item Object-Oriented Software Engineering Practical Software Development
	using UML and Java (second ed.), Lethbridge, Laganiere
	\item Head First Object Oriented Analysis and Design, McLaughlin, Pollice, West
	\item Head First Design Patterns, Freeman, Freeman 
\end{itemize}




\part{Paradigmes}



\chapter{Abstraction}
There are different meanning of abstractin. one of them is the ability to capture real world entities as classes.
Two types of abstractions in Java:
\begin{itemize}
	\item \textbf{interfaces}, used to define expected behaviour.Implementation \textbf{is hidden from a client}.
	\item \textbf{abstract classes}, used to define incomplete functionality.
\end{itemize}



\chapter{Inheritance}
The ability of subclass to derive members (fields and methods) from ascendands. In java only single parent class is allowed. It is an \textbf{'is-a'} relationship.
\textbf{Derived class inherits all members present in the base class. 
However not all of them are accessible}. This is ruled by access modifiers used in the base class.


\section{Accessing Members}
It is valid to instantiate an object with a subtype. The instantiated reference variable allows an access to those members (and their variations) which are present in their type.
It is still possible to access subtype members using cast:
\begin{verbatim}
Parent childParent = new Child();
\\ access to a member as it is defined in the Parent class.
childParent.field... 

\\ access to a member as it is defined in the Child class
((Child)childParent).field 
\end{verbatim}
\textbf{\textcolor{red}{Pay attention to the syntax of above cast!}}
%
%\begin{tabular}{i|l|l|l}
%	types & fields & cast & method calls \\
%	\hline
%	Parent pc = new Child() & parent & \textbf{to child} & \textbf{child \textcolor{red}{(*)}}\\
%	Parent p = new Parent() & parent & \textcolor{red}{\textbf{error!} & parent} \\
%	Child c = new Child() & child & \textbf{to parent} & child \\
%	Child c = new Parent() & \textcolor{red}{\textbf{error!}} & &\\
%	\hline
%\end{tabular}\\

\par
\textbf{\textcolor{red}{*  Those method which are overridden are accessible as usual.\\
In order to call methods that are not overridden the reference variable 
used to access the methods (here - \textit{pc}) must be cast to (Child)}}


\section{Pitfalls}

\subsection{Constructor Calls an Overridable Method}
\begin{enumerate}
	\item call to constructor in \textbf{child class}
	\item it calls \textbf{parent class constructor} first
	\item if there is a call to overridable method it calls \textcolor{red}{\textbf{textchild version of the method}}
	\item \textbf{ERROR! }The call \textcolor{red}{\textbf{will fail}} if the method references some uninitialised variable.
	\textcolor{red}{\textbf{The variable can be initialised only when the control returns to child constructor - in steps which will folow!}}
\end{enumerate}


\section{Liskov Substitution Principle}
Whenever an instance of some class is expected in a program, 
one can suply an instance  of subclass of the class



\chapter{Encapsulation}
Inner details of classes can be hidden by making them private and acceptible
through public API only - getters (accessors) and setters (mutators).



\chapter{Polymorphism , Method Overriding}
\textit{'Many forms'}. implemented by
\begin{itemize}
	\item subclass specialisation (\textit{is-a'} relationship
	\item Liskov substitution principle
	\item virtual method invocation)
\end{itemize}
Polymorphism is usually achieved by method overriding. It utilises method dynamic binding.


\section{Virtual method Invocation}
Method calls are dynamically dispatched based on runtime type of the receiver object.


\section{Method Overloading}
Overloading means that two or more methods have the same name
 but different signature. They \textbf{must have different parameters} (number of them and/or types),
 they \textbf{may have different return type and access modifiers} (private, protected, etc) - see Rules.
 
 
\section{Method Overriding}
It means that new implementation is provided to an inherited method. This is annotated by @Override. The overridden method in a superclass can be still called 
when invoked using \textbf{super.} keyword.

\subsection*{Rules}
\begin{itemize}
	\item \textbf{final, static, private} methods can't be overridden.
	\item access modifier of overriding method \textbf{must not be more restrictive}.
	\item \textbf{no new checked exception} can be thrown
	\item if return type is a reference type, then it can be original type or any descendant of this type (\textbf{covariant return type}).
\end{itemize}

\textbf{Private} methods can't be overriden because they are excluded from inheritance (not visible from within subclasses). \\
\textbf{Static} methods reside in static context, they belong to class, not to objects. Therefore they are not inherited, too. Hence, they can't be overriden. \\
However, they can be \textbf{shadowed} - subclass can have static method with the same name, as its parent class. A method call is bind to first method with a proper signature - starting 
from the class in current context and going up the inheritance tree.





\part{Design}



\chapter{Requiremments}


\section{Use Cases}


\end{document}