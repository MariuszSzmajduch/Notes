\documentclass{report}
\title{spark-notes}

\usepackage[utf8]{inputenc}
\usepackage{color}
\usepackage{hyperref}
\usepackage{tabto}

\begin{document}
\maketitle
\pagenumbering{gobble}
\tableofcontents




\part{Project Setup}



\chapter{Dependencies}

\section{Maven}
\begin{verbatim}
<dependency>
    <groupId>com.sparkjava</groupId>
    <artifactId>spark-core</artifactId>
    <version>2.7.2</version>
</dependency>
\end{verbatim}





\part{Server Operations}



\chapter{init()}
Default behaviour if it fails:
\begin{verbatim}
private Consumer<Exception> initExceptionHandler = (e) -> {
    LOG.error("ignite failed", e);
    System.exit(100);
};
\end{verbatim}

The behaviour can be customised by invoking 
\begin{verbatim}
initExceptionHandler(e -> ());
\end{verbatim}


\chapter{stop()}
Shuts down the server.




\part{Routes}



\chapter{Overview}
\begin{itemize}
	\item they are made of three peces:
	\begin{itemize}
		\item \textbf{a verb} - \textit{get, post put, delete, head, trace, connect, options}
		\item \textbf{a path} \textit{(URL)}
		\item \textbf{a callback} - \textit{(request, response) -\textgreater ()}
	\end{itemize}
	\item they are matched in the order they are defined
	\item \textbf{static import is recommended} for improved readability
\end{itemize}


\paragraph{get()} - shows something.
\paragraph{post()} - creates something
\paragraph{put()} - updates something
\paragraph{delete()} - self-explanatory
\paragraph{options()} - 'appease' something



\chapter{Named Paramaters}
\begin{itemize}
	\item preceded by a colon
	\item accessed by \textbf{\textit{params}} method on \textbf{\textit{request}} object. 
\end{itemize}
\begin{verbatim}
get("/path/:param", (req, res) -> {
    return "Hello " + req.params(":param");
});
\end{verbatim}


\section{splat()}
It works like a combination of wildcard and vararg. \textbf{\textit{splat()}} return an arrays of those parameters.
\begin{verbatim}
get("/path/*", (req, res) -> {
    return "Number of splat parameters is " + req.splat().length;
})
\end{verbatim}


\chapter{Grouping Routes}
We need to call a \textbf{\textit{path(..)}} method:

\begin{verbatim}
path("/api", () -> {
    before("/*", (q, a) -> log.info("Received api call"));
    path("/email", () -> {
        post("/add",       EmailApi.addEmail);
        put("/change",     EmailApi.changeEmail);
        delete("/remove",  EmailApi.deleteEmail);
    });

    path("/username", () -> {
        post("/add",       UserApi.addUsername);
        put("/change",     UserApi.changeUsername);
        delete("/remove",  UserApi.deleteUsername);
    });
});
\end{verbatim}




\part{Request}



\chapter{List of Methods}
request.attributes();        \tab 					    the attributes list \\
request.attribute("foo");  \tab 				   value of foo attribute\\
request.attribute("A", "V");		\tab 		  sets value of attribute A to V\\
request.body();           \tab 						 request body sent by the client\\
request.bodyAsBytes();   \tab 			  request body as bytes\\
request.contentLength();       \tab     length of request body\\
request.contentType();         \tab    		content type of request.body\\
request.contextPath();        \tab   		the context path, e.g. "/hello"\\
request.cookies();              \tab 	 	 request cookies sent by the client\\
request.headers();             \tab 			the HTTP header list\\
request.headers("BAR"	\tab 				value of BAR header\\
request.host    \tab 							the host, e.g. "example.com\\
request. \tab  										client IP address\\
request.params("foo"\tab 				value of foo path parameter\\
request.params(); \tab    					map with all parameters\\
request.pathInfo();          \tab  				the path info\\
request.port();        \tab 						the server port\\
request.protocol();           \tab    			the protocol, e.g. HTTP/1.1\\
request.queryMap();               \tab 			the query map\\
request.queryMap("foo");          \tab 		query map for a certain parameter\\
request.queryParams();            \tab 			the query param list\\
request.queryParams("FOO");       \tab 				value of FOO query param\\
request.queryParamsValues("FOO")  \tab 			all values of FOO query param\\
request.raw();                    \tab 									raw request handed in by Jetty\\
request.requestMethod();          \tab 				The HTTP method (GET, ..etc)\\
request.scheme();                 \tab 						"http"\\
request.servletPath();            \tab 					the servlet path, e.g. /result.jsp\\
request.session();                \tab 						session management\\
request.splat();                  \tab 							splat (*) parameters\\
request.uri();                    \tab 								the uri, e.g. "http://example.com/foo"\\
request.url();                    \tab 						the url. e.g. "http://example.com/foo"\\
request.userAgent();              \tab 					user agent\\




\part{Respond}



\chapter{List of Methods}

response.body();               \tab  		get response content\\
response.body("Hello");        \tab  		sets content to Hello\\
response.header("FOO", "bar"); \tab  		sets header FOO with value bar\\
response.raw();                \tab  		raw response handed in by Jetty
response.redirect("/example"); \tab  		browser redirect to /example\\
response.status();             \tab  		get the response status\\
response.status(401);          \tab  		set status code to 401\\
response.type();               \tab  		get the content type\\
response.type("text/xml");     \tab  		set content type to text/xml\\



\chapter{Querry Maps, Cookies, Sessions, Halting}
\url{http://sparkjava.com/documentation#sessions}


\section{Query Maps}
\begin{verbatim}
request.queryMap().get("user", "name").value();
request.queryMap().get("user").get("name").value();
request.queryMap("user").get("age").integerValue();
request.queryMap("user").toMap();
\end{verbatim}


\section{Cookies}
\begin{verbatim}
request.cookies();              // get map of all request cookies
request.cookie("foo"            // access request cookie by name
response.cookie("foo", "bar");        // set cookie with a value
response.cookie("foo", "bar", 3600);  // set cookie with a max-age
response.cookie("foo", "bar", 3600, true); // secure cookie
response.removeCookie("foo");              // remove cookie
\end{verbatim}



\section{Sessions}
\begin{verbatim}
// create and return session
request.session(true);                

 // Get session attribute 'user'
request.session().attribute("user");

// Set session attribute 'user'
request.session().attribute("user","foo"); 

// Remove session attribute 'user'
request.session().removeAttribute("user"); 
request.session().attributes();      // Get all session attributes
request.session().id();              // Get session id
request.session().isNew();           // Check if session is new
request.session().raw();             // Return servlet object
\end{verbatim}


\section{Halting}
Stops immediately a request within a filter or route.
\begin{verbatim}
halt();                // halt 
halt(401);             // halt with status
halt("Body Message");  // halt with message
halt(401, "Go away!"); // halt with status and message
\end{verbatim}


\chapter{Filters}


\section{Before}
They are evaluated \textbf{before each request}:
\begin{verbatim}
before((request, response) -> {
    boolean authenticated;

    // ... check if authenticated

    if (!authenticated) {
        halt(401, "You are not welcome here");
    }
});
\end{verbatim}

\section{After}
They are evaluated \textbf{after each request}:
\begin{verbatim}
after((request, response) -> {
    response.header("foo", "set by after filter");
});
\end{verbatim}


\section{afterAfter}
Works similar to 'finally' blocks:
\begin{verbatim}
afterAfter((request, response) -> {
    response.header("foo", "set by afterAfter filter");
});
\end{verbatim}


\section{Filters Taking Patterns}
They are evaluated \textbf{only if request path matches the pattern}:
\begin{verbatim}
before("/protected/*", (request, response) -> {
    // ... check if authenticated

    halt(401, "Go Away!");
});
\end{verbatim}




\part{TBC}
\url{http://sparkjava.com/documentation#response-transformer}



\end{document}