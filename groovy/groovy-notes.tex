\documentclass{report}
\title{Groovy Notes}
\author{msz}

\usepackage[utf8]{inputenc}
\usepackage{color}
\usepackage{hyperref}
\usepackage{breakurl}

\usepackage{tabto}

\begin{document}




\part{basics}



\chapter{Overview}


\begin{itemize}
  \item syntax very similar to java, and some fatures like in python
  \item optionally typed
\end{itemize}



\chapter{Syntax}


\section{def}
Tutorialspoint introduces it in a very messy way. Find another source explaining
exactly what is this.\\
A keyword used to define an identifier.\\
Can be also used as a method return type. But a method can be also void and
type-parametarised. So, what the difference????


\section{default parameters}
They provide a default value. \textbf{Must be listed after non-default
parameters}:
\begin{verbatim}
method(a, b, c - 3.0, d = "wortd"){
    ...
}
\end{verbatim}


\section{ranges}
Inclusive and exclusive, using nambers or characters:
\begin{verbatim}
a..b
a..<b   \\ exclusive
'd'..'l'
\end{verbatim}
There are builtin methods to operate on ranges:
contains(), get(), getFrom{}, getTo(), isReverse(), size(), subList().


\section{string literals - triple quotes}
Triple quotes can span multiple lines.


\section{lists}
\begin{verbatim}
[a, b, c]
\end{verbatim}
\begin{itemize}
  \item indexed
  \item standard list methods (add, get, size, contains, isEmpty, etc)
\end{itemize}


\section{regex}
\begin{verbatim}
~'^a.b?c?d{2}.[eaouyi]e*[c-m]*f+[0-9]+g$'
\end{verbatim}

\begin{itemize}
  \item a wildcard - \textbf{.}
  \item quantifiers:
  \begin{itemize}
    \item types:
    \begin{itemize}
      	\item \textbf{\{x\}} - exactly \textit{x} times
    	\item \textbf{?} - 0 or 1
    	\item \textbf{*} - 0 or many
    	\item \textbf{+} - 1 or many
    \end{itemize}
      
   	\item usage:
   	\begin{itemize}
     	\item applies to preceding character. 
     	\item applies to preceding set
   	\end{itemize}
   \end{itemize}
   
  \item sets
  \begin{itemize}
    \item explicit
    \begin{verbatim}
    [efh47]
    \end{verbatim}
    
    \item using ranges
    \begin{verbatim}
    [d-k3-8] 
    \end{verbatim}
    
    \item a quantifier can be applied after the set
    \item default - exactly 1 element when no quantifier applied.
  \end{itemize}

  \item \$ and \^{} denotes beginning and end of the line respectively
  \end{itemize}
\end{itemize}


\section{traits}
Not sure, but it looks like an interface, but with concrete methods and global
variables.
Then a trait can be implemented in the same way like a regular interface:
\begin{verbatim}
trait abc{
   var = 15
   traitMethod(){
   		doSomething();
   }
   
class MyClass implements abc{}
// and now var and the method doSomething are the parts of MyClass
\end{verbatim}


\section{closures}
The anonymous methods,  a lambdaexpression.
Can be used as parameters or assigned to variables. Mind the syntax for
parameter injected to a string and being standalone and that the closure is
called with \textbf{.call()}:
\begin{verbatim}
def closure = {a -> 10 + a};
def b = closure.call(2);
println(b);                 // prints '12'

// it's just a lambda but with 0 parameters
def closure2 = {println "Hi"};   
closure2.call();            // prints 'Hi'

// no brackets after println
// a parameter marked with $
def closure3 = {s -> println "Hi ${s}" };
closure3.call("Joe");        // prints 'Hi Joe'
\end{verbatim}


\subsection{usage with collections}
The method \textbf{.each} returns a stream of the collection items, similar to
foreach in java. They are passed to a closure as a parameter (multiple calls).
\begin{verbatim}
def list = [1, 2];
// this is 'x -> println(x)' under the hood
// x's are suplied by the each() method
list.each(println x);          
\end{verbatim}
This prints:\\
1\\
2




\part{XML}



\chapter(Generating XML)
MarkupBuilder library provides a support for XML:

\begin{verbatim}
import groovy.xml.MarkupBuilder 

class Example {
   static void main(String[] args) {
      def mp = [1 : ['Enemy Behind', 'War, Thriller','DVD','2003', 
         'PG', '10','Talk about a US-Japan war'],
         2 : ['Transformers','Anime, Science Fiction','DVD','1989', 
         'R', '8','A scientific fiction'],
         3 : ['Trigun','Anime, Action','DVD','1986', 
         'PG', '10','Vash the Stam pede'],
         4 : ['Ishtar','Comedy','VHS','1987', 'PG', 
         '2','Viewable boredom ']] 
			
      def mB = new MarkupBuilder()  
		
      // Compose the builder
      def MOVIEDB = mB.collection('shelf': 'New Arrivals') {
         mp.each {
            sd -> 
            mB.movie('title': sd.value[0]) {  
               type(sd.value[1])
               format(sd.value[2])
               year(sd.value[3]) 
               rating(sd.value[4])
               stars(sd.value[4]) 
               description(sd.value[5]) 
            }
         }
      }
   } 
}
\end{verbatim}

outputs:
\begin{verbatim}
<collection shelf = 'New Arrivals'> 
   <movie title = 'Enemy Behind'> 
      <type>War, Thriller</type> 
      <format>DVD</format> 
      <year>2003</year> 
      <rating>PG</rating> 
      <stars>PG</stars> 
      <description>10</description> 
   </movie> 
   <movie title = 'Transformers'> 
      <type>Anime, Science Fiction</type> 
      <format>DVD</format> 
      <year>1989</year>
	  <rating>R</rating> 
      <stars>R</stars> 
      <description>8</description> 
   </movie> 
   <movie title = 'Trigun'> 
      <type>Anime, Action</type> 
      <format>DVD</format> 
      <year>1986</year> 
      <rating>PG</rating> 
      <stars>PG</stars> 
      <description>10</description> 
   </movie> 
   <movie title = 'Ishtar'> 
      <type>Comedy</type> 
      <format>VHS</format> 
      <year>1987</year> 
      <rating>PG</rating> 
      <stars>PG</stars> 
      <description>2</description> 
   </movie> 
</collection> 
\end{verbatim}



\chapter{Parsing XML}
Supported by builtin \textbf{XmlParser} class (in groovy.util library) and its
\textbf{parse()} method.
Using XML files generated in previous chapter:
\begin{verbatim}
import groovy.xml.MarkupBuilder 
import groovy.util.*

class Example {

   static void main(String[] args) { 
	
      def parser = new XmlParser()
      def doc = parser.parse("D:\\Movies.xml");
		
      doc.movie.each{
         bk->
         print("Movie Name:")
         println "${bk['@title']}"
			
         print("Movie Type:")
         println "${bk.type[0].text()}"
			
         print("Movie Format:")
         println "${bk.format[0].text()}"
			
         print("Movie year:")
         println "${bk.year[0].text()}"
			
         print("Movie rating:")
         println "${bk.rating[0].text()}"
			
         print("Movie stars:")
         println "${bk.stars[0].text()}"
			
         print("Movie description:")
         println "${bk.description[0].text()}"
         println("*******************************")
      }
   }
}
\end{verbatim}

This produces:
\begin{verbatim}
Movie Name:Enemy Behind 
Movie Type:War, Thriller 
Movie Format:DVD 
Movie year:2003 
Movie rating:PG 
Movie stars:10 
Movie description:Talk about a US-Japan war 
******************************* 
Movie Name:Transformers 
Movie Type:Anime, Science Fiction 
Movie Format:DVD 
Movie year:1989 
Movie rating:R 
Movie stars:8 
Movie description:A schientific fiction 
******************************* 
Movie Name:Trigun 
Movie Type:Anime, Action
Movie Format:DVD 
Movie year:1986 
Movie rating:PG 
Movie stars:10 
Movie description:Vash the Stam pede! 
******************************* 
Movie Name:Ishtar 
Movie Type:Comedy 
Movie Format:VHS 
Movie year:1987 
Movie rating:PG 
Movie stars:2 
Movie description:Viewable boredom
\end{verbatim}
 
 
 
\part{JSON}
Groove contains builtin libraries supporting JASON parsing and creation:
\begin{itemize}
  \item \textbf{JsonSlurper} class supports parsing. Methods:
  \begin{itemize}
    \item \textbf{parseText()}. We need JsonSLurper instance to use it.
  \end{itemize}
  
  \item \textbf{JsonOutput} class supports generating. Methods:
  \begin{itemize}
    \item static JsonOutput.\textbf{toJson}()
  \end{itemize}
\end{itemize}



\part{resurces}



This one is terrible poor, try to find something else:\\
\url{https://www.tutorialspoint.com/groovy/index.htm}


\end{document}